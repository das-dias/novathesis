%!TEX root = ../template.tex
%%%%%%%%%%%%%%%%%%%%%%%%%%%%%%%%%%%%%%%%%%%%%%%%%%%%%%%%%%%%%%%%%%%%
%% abstract-en.tex
%% NOVA thesis document file
%%
%% Abstract in English([^%]*)
%%%%%%%%%%%%%%%%%%%%%%%%%%%%%%%%%%%%%%%%%%%%%%%%%%%%%%%%%%%%%%%%%%%%

\typeout{NT FILE abstrac-en.tex}%

\textit{Vagus nerve stimulation} (VNS) is an effective therapeutic medical procedure 
for the supression of epileptic seizures, depression, mood and apetite control and 
chronic pulmonary obstructive disease (COPD).
Low-intensity focused ultrasound (LIFUS) has been shown to be a promising 
non-invasive neuromodulation technique to perform VNS.
The safety and effectiveness of the procedure is ensured by
trained physicians and technitians operating the device while relying on real-time 
imaging to guide the procedure. Failing to acquire the target nerve results in 
poorly applied US sonication procedure, reducing the effectiveness of the therapeutic session. 
Mechanical or thermally-induced cellular tissue damage can also become evermore 
likely when aiming for finer and more precise stimulation procedures. 

The need for an automated closed-loop control system is evident to ensure 
the focal spot is correctly positioned, ensuring optimal application of the 
stimulation proceudre. Correct application of US sonication protocols not 
only has the potential to reduce therapeutic sessions duration, but also 
enabling improved exploratory medical procedures to study the regulatory function 
of the peripheral nervous system by reducing the number of control variables 
during scientific experimental procedures.

Integrating the acquisition of the target nerve's position closer to the sensor 
ultimately leads to a lower power and highly portable medical device that can evolve 
to a point-of-care device.
This dissertation proposes a CMOS application-specific integrated circuit (ASIC) for 
real-time US imaging, integrating a capacitive gated recurrent unit (GRU) neural network
within its readout signal chain capable of closing the loop of a control system aiming 
for automatic, secure and precise correction of the beamsteering procedure. 

% Palavras-chave do resumo em Inglês
% \begin{keywords}
% Keyword 1, Keyword 2, Keyword 3, Keyword 4, Keyword 5, Keyword 6, Keyword 7, Keyword 8, Keyword 9
% \end{keywords}
\keywords{
  One keyword \and
  Another keyword \and
  Yet another keyword \and
  One keyword more \and
  The last keyword
}