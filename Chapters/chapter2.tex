%!TEX root = ../template.tex
%%%%%%%%%%%%%%%%%%%%%%%%%%%%%%%%%%%%%%%%%%%%%%%%%%%%%%%%%%%%%%%%%%%%
%% chapter2.tex
%% NOVA thesis document file
%%
%% Chapter with the template manual
%%%%%%%%%%%%%%%%%%%%%%%%%%%%%%%%%%%%%%%%%%%%%%%%%%%%%%%%%%%%%%%%%%%%

\typeout{NT FILE chapter2.tex}%

\chapter{Ultrasound, Transducers and Ultrasound Imaging}
\label{cha:users_manual}

\glsresetall




\section{Ultrasound Fundamentals}
\label{sec:ultrasound_fundamentals}

\section{Ultrasound Transducers}
\label{sec:ultrasound_transducers}

\section{Ultrasound Imaging}
\label{sec:ultrasound_imaging}

The pressure generated at the surface of the elements depends on the amplitude of the applied pulse, and the transmit efficiency (expressed in Pa/V) of the transducer. In the case of focused transmission, the pulses are timed by the transmit beamformer such that the gen- erated acoustic waves converge at the desired focal point. The pressure at this focal point can then be found by taking into account the TX beamforming gain and the propagation attenuation in the medium. 

The acoustic wave will reflect from interfaces between regions with different acoustic impedance in the medium, leading to echoes that return to the transducer. The amplitude of an echo depends on the reflection coefficient associated with the interface, and the geometrical spreading and atten- uation that the echo experiences as it travels back to the transducer. The resulting surface pressure at one of the elements then translates to a voltage through the trans- ducer’s receive sensitivity (expressed in V/Pa). This signal is amplified by the LNA and TGC and digitized, and finally combined with the signals from other elements by the receive beamformer. Upon performing delay-and-sum (DAS), the correlated signals from the different channels add up constructively, while uncorrelated noise does not, giving an RX beamforming gain and leading to the creation of a pressure profile through time, leading to the reconstruction of an US B-mode image upon quadrature-filtering and log-compression of the beamformed RX US signals.



\section{Compressive Sensing - Emerging Ultrasound Imaging Algorithms}